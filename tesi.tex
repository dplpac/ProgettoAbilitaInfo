\documentclass[a4paper,12pt,titlepage]{article}
\pdfpagewidth
\paperwidth
\pdfpageheight
\paperheight



\usepackage[italian]{babel} 
\usepackage[T1]{fontenc} 
\usepackage[utf8]{inputenc} 
\usepackage{graphicx,color,listings}
\usepackage{fancyhdr} 
\usepackage{mathtools}
\usepackage{mhsetup}
\usepackage{amscd} 
\usepackage[usenames,dvipsnames]{xcolor}
\frenchspacing
\usepackage{geometry}
\usepackage{caption}
\captionsetup{labelformat=empty, textfont=sl}
\geometry{a4paper,tmargin=3cm,bmargin=3cm, lmargin=3cm,rmargin=2cm} \usepackage{multirow}

\textwidth16cm\textheight24cm\topmargin0mm\headheight0mm\headsep6mm\oddsidemargin0mm\evensidemargin0mm

\usepackage{siunitx}



\begin{document}
\author{David De Pol}
\title{Simulazione di scenari di scuotimento
sismico ad Ischia mediante l'utilizzo
della Web App XeRiS}
\maketitle
\tableofcontents
\clearpage

\section{Introduzione}

\section{Presentazione della Web App XeRiS}
\subsection{Definizione del modello strutturale}
\subsection{Modellizzazione di sorgenti sismiche estese}
\subsection{Modellazione del moto del suolo generato da sorgenti sismiche
estese}
\subsection{Parametri di scuotimento del suolo}
\subsection{Test parametrici}

\section{Simulazione di eventi di scenario}
\subsection{Simulazione del terremoto a Ischia: Faglia di Casamicciola}
\subsection{ Simulazione di eventi di scenario ad Ischia per sorgenti appenniniche}
\subsection{Confronto tra alcune simulazioni con sorgente appenninica}

\section*{Conclusioni}\addcontentsline{toc}{section}{Conclusioni}

\end{document}